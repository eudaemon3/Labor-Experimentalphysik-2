\documentclass{article}
\usepackage[german]{babel}
\usepackage[utf8]{inputenc}
\usepackage[a4paper,top=2cm,bottom=2cm,left=3cm,right=3cm,marginparwidth=1.75cm]{geometry}
\usepackage{amsmath}
\usepackage[onehalfspacing]{setspace}
\usepackage{graphicx}
    \graphicspath{ {./images/} }
\usepackage{url}
\usepackage{float}
\usepackage{pdfpages}
\usepackage{lastpage}
\usepackage{siunitx}
\sisetup{separate-uncertainty=true}
\usepackage{gensymb}
\usepackage{indentfirst}
\usepackage[square,numbers]{natbib}
\usepackage[colorlinks=true, allcolors=blue]{hyperref}
\usepackage[headsepline]{scrlayer-scrpage}
\pagestyle{scrheadings}
\clearpairofpagestyles
\begin{document}
\setcounter{totalnumber}{4}
\includepdf{Deckblatt.pdf}

\ihead{Brandstötter, Brötz}
\chead{}
\ohead{}
\cfoot{\thepage{} / \pageref{LastPage}}

\tableofcontents
\newpage

\section{Auswertung}

\begin{figure}[h!]
    \centering
    \includegraphics[width=0.8\linewidth]{images/Messung_V1_StromSpannung.png}
    \caption{Aufgezeichnete zeitliche Verläufe der Spannungsabfälle über den Widerstand $R$ und den Kondensator $C$ bei einer sinusförmigen Speisespannung in der Schaltung aus Versuch 1. }
    \label{fig:MessungKondensatorSpannung}
\end{figure}

\begin{table}[h]
    \centering
    \caption{Gemessene Gesamtwiderstände $R_\text{ges}$ inklusive Messunsicherheit dargestellt für die gewählten Potentiometer-Stellungen zur Erzeugung der drei verschiedenen Spannungsverläufe.}
    \label{tab:MessungenWiderstand}
    \begin{tabular}{c|c|c}
         - & \textbf{Widerstand} $R_\text{ges}$ / $\Omega$ & \textbf{Unsicherheit} $\Delta R_\text{ges}$ / $\Omega$ \\\hline
         Kriechfall &  $530$ & $5$ \\ 
         Grenzfall   &  $346$ & $4$ \\
         Schwingfall   &  $85.7$ & $1.0$ \\\hline
    \end{tabular}
\end{table}

\section{Diskussion}
\bibliographystyle{abbrvnat}
\bibliography{lit.bib}
\newpage
\listoffigures
\listoftables
\end{document}

 1 & $140{,}0 \pm 0{,}5$ & $188\pm 6$ \\
 2 & $130{,}0 \pm 0{,}5$ & $189\pm 6$ \\
 3 & $120{,}0 \pm 0{,}5$ & $177\pm 5$ \\
 4 & $110{,}0 \pm 0{,}5$ & $166\pm 5$ \\
 5 & $100{,}0 \pm 0{,}5$ & $140\pm 5$ \\
 6 & $ 90{,}0 \pm 0{,}5$ & $ 95\pm 3$ \\
 7 & $ 80{,}0 \pm 0{,}5$ & $ 72\pm 3$ \\
 8 & $ 70{,}0 \pm 0{,}5$ & $34{,}7 \pm 1{,}4$ \\
 9 & $ 60{,}0 \pm 0{,}5$ & $10{,}6 \pm 0{,}8$ \\
10 & $ 50{,}0 \pm 0{,}5$ & $ 0{,}7 \pm 0{,}6$ \\
11 & $ 40{,}0 \pm 0{,}5$ & $ 3{,}9 \pm 0{,}6$ \\
12 & $ 30{,}0 \pm 0{,}5$ & $21{,}8 \pm 1{,}1$ \\
13 & $ 20{,}0 \pm 0{,}5$ & $51{,}7 \pm 1{,}8$ \\
14 & $ 10{,}0 \pm 0{,}5$ & $ 86\pm 3$ \\
15 & $  0{,}0 \pm 0{,}5$ & $125\pm 4$ \\
16 & $350{,}0 \pm 0{,}5$ & $135\pm 4$ \\
17 & $340{,}0 \pm 0{,}5$ & $176\pm 5$ \\
18 & $330{,}0 \pm 0{,}5$ & $188\pm 6$ \\
19 & $320{,}0 \pm 0{,}5$ & $199\pm 6$ \\